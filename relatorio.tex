%%%%%%%%%%%%%%%%%%%%%%%%%%%%%%%%%%%%%%%%%%%%%%%%%%%%%%%%%%%%%%%
%
% Welcome to Overleaf --- just edit your LaTeX on the left,
% and we'll compile it for you on the right. If you open the
% 'Share' menu, you can invite other users to edit at the same
% time. See www.overleaf.com/learn for more info. Enjoy!
%
%%%%%%%%%%%%%%%%%%%%%%%%%%%%%%%%%%%%%%%%%%%%%%%%%%%%%%%%%%%%%%%
\documentclass[titlepage,12pt,a4paper]{article}
%encoding
%--------------------------------------
\usepackage[T1]{fontenc}
\usepackage[utf8]{inputenc}
%--------------------------------------
%Portuguese-specific commands
%--------------------------------------
\usepackage[portuguese]{babel}
%--------------------------------------
\usepackage{todonotes}
\usepackage{listings}
\usepackage{tabularx}
\usepackage{float}
\usepackage{caption}
\captionsetup[figure]{font=footnotesize,labelfont=bf}
%--------------------------------------
\makeatletter
\renewcommand*\l@section{\@dottedtocline{1}{0em}{1.5em}}
\makeatother
\hyphenchar\font=-1

\begin{document}
    \begin{titlepage}
        \centering
        {\LARGE PSI3471 - Sistemas eletrônicos inteligentes \par} 
        \vfill
        {\Huge\bfseries EP1 \par}
        \vfill
        \raggedleft
        {\large Tiago Azevedo Amano \qquad 5974802 \par}
        \vspace{2.5cm}
        \centering
        {\today\par}
    \end{titlepage}
    \tableofcontents
    \vspace{2cm}
    \hline
    \section{Enunciado}
    \label{sec:enunciado}
    O objetivo deste exercício é fazer um programa C/C\texttt{++} que, dada uma imagem de moedas brasileiras atuais sobre fundo de papel azul, conta quantas moedas tem na imagem e localiza espacialmente cada moeda. Opcionalmente, o programa pode identificar o valor de cada moeda e calcular o valor total das moedas na imagem.\\\\
    A forma de chamar o programa deve ser obrigatoriamente:
    \begin{lstlisting}
        linux$ ep1 moeda1.jpg saida1.png
        windows> ep1 moeda1.jpg saida1.png
    \end{lstlisting}
    O programa deve emitir um aviso amigável se o usuário entrar argumentos inválidos (número de parâmetros incorreto, arquivo inexistente, etc).\\\\
    \section{Resolução}
    \label{sec:resolução}
    \begin{itemize}
        \item Primeiramente, reduzi a imagem em 5 vezes (de 3264x2448 para 652x489) para melhorar o desempenho em tempo de execução e no reconhecimento das moedas.
        \begin{figure}[H]
            \centering
            \includegraphics[width=0.75\textwidth]{imgreduz.jpg}
            \caption{\footnotesize Imagem moeda4.jpg reduzida de 5 vezes}
            \label{fig:fig1}
        \end{figure}
        \item Converti a imagem de BGR para HSV pois o parâmetro \textit{Hue} tem uma separação razoável entre as moedas e o fundo azul.
        \item Adaptei o algoritmo apresentado em aula pelo prof. Hae %(pintafila2.cpp)inserir aqui referencia%
        para pintar usando fila os pixels do fundo da imagem de branco (tomei como pixels de fundo todos os que tinham valor de \textit{Hue} maiores que 30)
        \item Pintei de preto as moedas utlizando componentes conexos, i.e. varri a imagem procurando pixels que não fossem do fundo (i.e. pretos) ou já pintados (i.e. brancos). Quando encontrado, o pixel é parte de um componente conexo, e então pinto a partir desse pixel todos os pixels deste componente conexo, fazendo isso para todos os pixels da imagem. Mostrado na Figura \ref{fig:fig1}\\
        \begin{figure}[H]
            \centering
            \includegraphics[width=0.75\textwidth]{tmp.jpg}
            \caption{\footnotesize Imagem moeda4.jpg com moedas pintadas de preto e fundo de branco}
            \label{fig:fig1}
        \end{figure}
        \item Converti a imagem de HSV de volta para BGR.
        \item Criei 8 modelos de círculo, pretos por dentro e brancos por fora com diferentes escalas separadas por progressão aritmética.
        \item Procurei todos os modelos de diferentes escalas na imagem, gerando uma imagem com os valores de correlação em ponto flutuante para cada modelo.
        \item Converti as 8 imagens com os valores de correlação em ponto flutuante para escala de cinza para poder trabalhar mais facilmente, dado que o programa de edição/visualização de imagem que utilizei me dá os valores das cores pixels de 0 a 255 em RGB.
        \item Criei uma imagem em BGR consolidando as 8 imagens em uma só com os valores de correlação como o primeiro parâmetro, um valor de 0 a 7 representando a escala do modelo com o maior valor de correlação em um dado pixel como o segundo parâmetro e o valor 0 como o terceiro parâmetro. Mostrado na Figura \ref{fig:fig2}.
        \begin{figure}[H]
            \centering
            \includegraphics[width=0.75\textwidth]{temax.jpg}
            \caption{\footnotesize Valores de correlação e escala do modelo com maior correlação para a imagem moeda4.jpg}
            \label{fig:fig2}
        \end{figure}
        \item Criei uma lista com todos os pixels candidatos a centro, i.e. os pixels que tem valor de correlação maior que um limiar (no caso, o limiar é 190) e que são maiores ou iguais aos pixels vizinhos.
        \item Eliminei as falsas detecções eliminando os pixels-candidatos que estão muito próximos entre si (usei como parâmetro a distância euclidiana menor que 15 pixels).\\\\
        Para isso, crio uma cópia da lista e testo cada item da lista original com a segunda lista e toda vez que a distância entre dois pixels-candidatos da lista era menor que 15, coloco o com maior correlação no fim da lista ou uma média dos pixels, da correlação e da escala caso tiverem mesmo valor de correlação para testar novamente caso haja mais pixels-candidatos próximos. Caso a distância euclidiana for maior que 15, coloco o pixel numa lista separada para os centros garantidos.
        \item Conto o número de elementos da lista de centros para saber o número de moedas, escrevo na imagem reduzida o número de moedas que está nela e desenho na imagem reduzida circulos das moedas com tamanhos do modelo com maior número de correlação para cada centro da lista centrados nesses centros e um ponto no centro finalmente volto a imagem reduzida para a resolução da imagem original.
        \begin{figure}[H]
            \centering
            \includegraphics[width=0.75\textwidth]{teste.jpg}
            \caption{\footnotesize Imagem com a contagem e localização das moedas para a imagem moeda4.jpg}
            \label{fig:fig2}
        \end{figure}
    \end{itemize}
    \begin{lstlisting}
    \end{lstlisting}
    \section{Ambiente de desenvolvimento}
    \label{sec:ambiente}
    Desenvolvi o programa no linux, utilizando as bibliotecas cekeikon5.6 e queue.\\
    Para compilar o programa, pode utilizar o comando \textit{\bfseries compila} com o parâmetro \textit{\bfseries -cek}:
    \begin{lstlisting}
    $ compila ep1.cpp -cek
    \end{lstlisting}
    \\\\
    \section{Operação}
    \label{sec:operação}
    Para executar o programa, deve chamar no terminal ou no prompt de comando:
    \begin{lstlisting}
    $ ep1 imagem-de-entrada.jpg imagem-de-saida.jpg
    \end{lstlisting}
    Onde a imagem de entrada deve ser uma imagem com pelo menos uma moeda de 1 real e os formatos das imagens de entrada e saída devem ser os mesmos.\\
    São gerados, além da imagem-de-saída.jpg, as imagens intermediárias apresentadas neste relatório, como a imagem de entrada reduzida, a imagem reduzida com fundo branco e moedas em preto e a imagem com os valores de correlação e escala.\\
    \section{Resultados}
    \label{sec:resultados}
    Os tempos de execução, tirando média de 10 execuções usando WSL (bash ubuntu no windows):
    \begin{itemize}
        \item Para imagem moeda1.jpg: 1.036s
        \item Para imagem moeda2.jpg: 1.011s
        \item Para imagem moeda3.jpg: 1.104s
        \item Para imagem moeda4.jpg: 1.091s
    \end{itemize}
    Média para todas as execuções: 1.060s\\\\
    Em todas as execuções do programa foi contada corretamente o número de moedas, com pequeno erro na localização, porém satisfatório
\end{document}